\pdfoutput=1
\documentclass[pra,twocolumn,showpacs,amsmath,amssymb]{revtex4-1}

\usepackage{graphicx}%Include figure files
\usepackage{dcolumn}%Align table columns on decimal point
\usepackage{bm}% bold math

%\nofiles

\begin{document}

\title{Project 1: Modelling Radioactive Decay}


\author{John Russo}
\affiliation{Department of Physics and Astronomy, University
of Delaware, Newark, DE 19716-2570, USA}

\begin{abstract}
Abstract goes here.
\end{abstract}

\pacs{}


\maketitle

\section{Introduction} \label{sec:intro}

Modelling radioactive decay involves only a simple numerical solution to a
differential equation. However, it is a useful demonstration of basic numerical
solving techniques, with a wide range of applications.
Radioactive decay is a stochastic process where a nucleus breaks down, releasing
energy, matter, or both, depending on the type of decay.
Radioactive decay is a process studied for applications from nuclear weapons to
carbon dating, where understanding decay rates of isotopes can yield information
about age of a sample.

\section{Method}

\section{Results}

For all results presented here, $N_A = 200, N_B = 5, \tau_A = 0.5$ and $\tau_B = 1.0$.

\begin{figure}
  \vspace{2ex}
  \includegraphics[width=\linewidth]{../plot.pdf}
  \caption{A and B particle populations, plotted against arbitrary time. Using
  a shared linear axis, the large difference in population sizes makes
  visualization of the data difficult.}
  \label{fig:linplot}
\end{figure}

\begin{figure}
  \vspace{2ex}
  \includegraphics[width=\linewidth]{../logplot.pdf}
  \caption{A and B particle populations, plotted against arbitrary time. The log
  scale shows that though the populations decay at very different rates, the
  linearity of each population's trajectory demonstrates exponential decay.}
  \label{fig:logplot}
\end{figure}

In these figures, and particularly in Fig.~\ref{fig:linplot} one can observe the
B particle population sharply peaking before beginning its decline. When the
A population satisfies
\begin{equation}
\frac{N_A}{N_B} \geq \frac{\tau_A N_{B,0}}{\tau_B N_{A,0}}
\end{equation}
the growth rate of the B population is positive. For the given initial parameters,
this means the B population should start declining when $N_A/N_B \geq 0.0125$.

Observing the turnaround time here to be when the $N_B \approx 4000$,
we can see in Fig.~\ref{fig:logplot} that at that time $N_A \approx 50$.

\section{Conclusions} \label{sec:conclusion}

\begin{thebibliography}{1}

\end{thebibliography}

\end{document}
